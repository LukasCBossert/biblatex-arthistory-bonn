% arthistory-bonn --%
% Copyright (c) 2017 Lukas C. Bossert | Thorsten Kemper
%  
% This work may be distributed and/or modified under the
% conditions of the LaTeX{} Project Public License, either version 1.3
% of this license or (at your option) any later version.
% The latest version of this license is in
%   http://www.latex-project.org/lppl.txt
% and version 1.3 or later is part of all distributions of LaTeX
% version 2005/12/01 or later.
% !TEX program = lualatex
\documentclass[a4paper,
10pt,
ngerman,
english
]{ltxdoc}
\input{arthistory-bonn-preamble.tex}

\begin{document}
\title{\texttt{arthistory-bonn} -- \\\texttt{bib\LaTeX} for art historians\footnote{The development of the code is done at \url{https://github.com/LukasCBossert/biblatex-arthistory-bonn}. 
Comments and criticisms are welcome.}}

\author{Lukas C. Bossert\protect\\%
{\small \href{mailto:lukas@digitales-altertum.de}{lukas@digitales-altertum.de}}
 \and Thorsten Kemper}
 
\date{Version: \arthistoryversion\ (\arthistorydate)
} 
\maketitle
 
 \begin{abstract}
\noindent This citation style covers the citation and bibliography guidelines of the \foreignquote*{ngerman}{Kunsthistorisches Institut der Universität Bonn} for undergraduates.
It introduces bibliography entry types for catalogs and features a tabular bibliography, among other things.
Various options are available to change and adjust the outcome according to one's own preferences. 
The style is compatible with English and German.
 \end{abstract}


\begin{multicols}{2}
\footnotesize\parskip=0mm \tableofcontents
\end{multicols}

\section{Introduction}
|arthistory-bonn| is a bib\LaTeX{} style that complies with the guidelines of the Institute of History of Art at the University of Bonn, or \foreignquote*{ngerman}{Kunsthistorisches Institut der Universität Bonn}\footnote{Website: \url{https://www.khi.uni-bonn.de}.} (henceforth KHI), as laid out in the compendium \foreignquote*{ngerman}{Aufbau und Gestaltung von Seminararbeiten} for first-year students of art history. In particular, it introduces
\begin{itemize}
	\item new entry types for exhibition catalogs and inventory catalogs, and
	%\item special citation short forms for catalogs, encyclopediae as well as primary sources, and
	\item a tabular bibliography that lists each entry's citation short form and its full citation, sorted by the citation short forms.\footnote{Strictly speaking, the solution presented here is a mix of the two bibliography styles proposed in the guideline: While our implementation does not technically involve \LaTeX{} tabular environments, for bibliography entries with short labels, the entry will appear \emph{as} in a table, i.e. the label/short citation will be listed in the left \enquote{column}, and the bibliography entry will begin on the same line in the right \enquote{column}. Entries with a label longer than the width of the left column will see the actual bibliography entry in the next line however, yet aligned to the right column as well. This way the citation short forms will not have to break, leading to a cleaner page, we believe.}
\end{itemize}
The special citation short forms for catalogs, encyclopediae as well as primary sources described in the KHI's guideline are being achieved by using |bib|\LaTeX's standard bibliography entry options.

Various options are available to adjust to common practices not covered by the KHI's advised rules.
%Fortsetzen


\section{Installation}
|arthistory-bonn| is part of the distributions MiK\TeX{} and \TeX{} Live and you can easily install it using the respective package manager. 
If you would like to install |arthistory-bonn| manually, do the following:
Download the folder |arthistory-bonn| with all relevant files from the CTAN-server\footnote{\url{http://mirrors.ctan.org/macros/latex/contrib/biblatex-contrib/arthistory-bonn.zip}} and copy it to the \texttt{\$LOCALTEXMF} directory of
 your system.\footnote{If you don't know what that is, have a look at
\url{http://www.tex.ac.uk/cgi-bin/texfaq2html?label=tds} or 
\url{http://mirror.ctan.org/tds/tds.html}.} 
Refresh your filename database.\footnote{ 
Here is some additional information from the UK \TeX\ FAQ:
\begin{itemize}[nosep,after=\vspace{-\baselineskip} ]
  \item \href{%
    http://www.tex.ac.uk/cgi-bin/texfaq2html?label=install-where}{%
    Where to install packages}
  \item \href{%
    http://www.tex.ac.uk/cgi-bin/texfaq2html?label=inst-wlcf}{%
    Installing files \enquote{where \LaTeX /\TeX\ can find them}}
  \item \href{%
    http://www.tex.ac.uk/cgi-bin/texfaq2html?label=privinst}{%
    \enquote{Private} installations of files}
\end{itemize}
}


\section{Loading the package}
 \DescribeMacro{arthistory-bonn} The name of the bib\LaTeX style is |arthistory-bonn|. It has to be activated in the preamble.

\begin{code}
\usepackage[style=arthistory-bonn,%
          *@\meta{further options}@*]{biblatex}
\bibliography*@\marg{|bib|-file}@*
\end{code}

Without enabling any further options, the style follows the rules of the \foreignquote*{ngerman}{Kunsthistorisches Institut der Universität Bonn}. No additional settings are needed, but you can change the outcome by using a couple of options which are explained below.

At the end of your document you can write the command |\printbibliography| to print 
a single bibliography.
However, since |arthistory-bonn| supports citation styles for catalogs and primary sources that differ from the standard citations of common scientific contributions, you can list these entry types in separate bibliographies, see \cref{sec:sepbib}.


\section{Bibliography entries}
Besides loading |arthistory-bonn|, in order to comply with the KHI guideline's bibliography rules, users' actions will be required mostly when entering bibliography items.

\subsection{Entry fields}

\subsubsection{\texttt{arthistory-bonn}-specific options}

\DescribeMacro{arthist}
\enquote{H-ArtHist} is a popular email newsletter for art historians, publishing, among other things, reviews. These reviews share very similar URLs, and |arthist=|\marg{value} will print the URL of the respective piece, where the \meta{value} must be read off the respective URL.
\DescribeMacro{arthistdate}
Relatedly, |arthistdate={|\meta{year}|-|\meta{month}|-|\meta{day}|}| specifies the release date of the newsletter issue. A typical entry may contain entries such as
\begin{code}
@Review{vonEngelberg2003,
  *@\ldots@*
  arthist      = {35},
  arthistdate  = {2003-11-12},
  *@\ldots@*
}.
\end{code}

\DescribeMacro{eventsubtitle}
Use this field to specify the subtitle of an exhibition a given |exhibcatalog| (see \cref{sec:arth-etypes}) is based on. See also |eventtitle| in \cref{sec:bibl-efields}.

\DescribeMacro{exhibfirstdate}
Specifies the time span of an exhibition an |exhibcatalog| is based on. Dates of the first and last day of the respective exhibition are to be entered in the format \meta{year}|-|\meta{month}|-|\meta{day}. A typical entry may be
\begin{code}
@Exhibcatalog{AusstellungBonn2005,
  *@\ldots@*
  exhibfirstdate = {2005-11-25/2006-03-19},
  *@\ldots@*
}.
\end{code}
\DescribeMacro{exhibseconddate,\\exhibthirddate}If an exhibition has more than one date, use |exhibseconddate| and |exhibthirddate| accordingly.

\DescribeMacro{exhibfirstlocation}
For the city in which an exhibition has been shown; e.\,g.,
\begin{code}
@Exhibcatalog{AusstellungBonn2005,
  *@\ldots@*
  exhibfirstlocation = {Bonn},
  *@\ldots@*
}.
\end{code}
\DescribeMacro{exhibsecondlocation,\\exhibthirdlocation}If an exhibition has more than one date, use |exhibsecondlocation| and |exhibthirdlocation| accordingly.

\DescribeMacro{exhibfirstmuseum}
For the venue---usually a museum---where an exhibition is shown; e.\,g.,
\begin{code}
@Exhibcatalog{AusstellungBonn2005,
  *@\ldots@*
  exhibfirstmuseum = {Kunst- und Ausstellungshalle der Bundesrepublik Deutschland},
  *@\ldots@*
}.
\end{code}
\DescribeMacro{exhibsecondmuseum,\\exhibthirdmuseum
}If an exhibition has more than one date, use |exhibsecondmuseum| and |exhibthirdmuseum| accordingly.

\DescribeMacro{thesisdate}
The year when a PhD (or \enquote{Habilitation}) thesis was defended.

\DescribeMacro{thesistype=\{tzugl\}}
To be used when a publication (typically a book) is \emph{only partly} based on a submitted PhD (or \enquote{Habilitation}) thesis.


\subsubsection{Important standard \texttt{bib}\LaTeX{} options}\label{sec:bibl-efields}
Here, we list otherwise standard |bib|\LaTeX{} options that are essential to comply with the KHI's bibliography rules.

\DescribeMacro{eventdate}
Use this field to specify the year(s) of an exhibition a given |exhibcatalog| is based on. In case the exhibition covered two subsequent years, enter them as \meta{first year}|/|\meta{second year}. An example would be
\begin{code}
@Exhibcatalog{AusstellungBonn2005,
  *@\ldots@*
  eventdate = {2005/2006},
  *@\ldots@*
}.
\end{code}
<<<<<<< HEAD
Note that specifying this field has the sole purpose of generating the correct citation short form.
You will also have to enter the field |exhibfirstdate| (and possibly |exhibseconddate| or even |exhibthirddate|) for the long bibliography entry, and the fields |date| or |year| for the publication itself. 
=======
Note specifying this field has the sole purpose of generating the correct citation short form.
You will also have to enter the field |exhibfirstdate| (and possibly |exhibseconddate| or even |exhibthirddate|) for the long bibliography entry, 
and the fields |date| or |year| for the publication itself. 
>>>>>>> 4f0f6faf012d1bb030313f799018ae047978aa94

\DescribeMacro{eventtitle}
Use this field to specify the name of an exhibition a given |exhibcatalog| is based on.

\DescribeMacro{institution}
For the institution at which a thesis was defended.

\DescribeMacro{keywords=\{source\}}
%\DescribeMacro{source}
This option is reserved for entries that are primary sources (e.\,g. Alberti, Paleotti etc). %cf. \cref{source}.
If enabled, the entry can be listed in a separate bibliography for primary sources. 
(Actually you don't need to use the term \enquote*{source} -- you can pick any term you like.) 
Also see \cref{sec:sepbib}.

In addition to that, you should define a |shorthand| that differs from the usual author-year citation of \enquote*{regular}, 
modern scientific works in order to comply with the KHI's citation rules.

\DescribeMacro{shorthand}\label{sec:shorthand}
%\DescribeMacro{shorthand}
Entering a shorthand will replace the otherwise automatically generated, document type-appropriate citation short form by the typed-in content.

You need to make use of this option when entering a primary source in your bibliography file (along with the option |keywords=\{source\}|). 
The entry should consist of a short version of the primary source's author's name and (possibly an abbreviation of) their contribution's title.

Here is an example of Casanova's Theory of painting:
\begin{bibexample}[label=CasanovaMalerei]{{@}Book\{CasanovaMalerei,…\}}
@Book{CasanovaMalerei,
  author    = {Casanova, Giovanni Battista},
  editor    = {Kanz, Roland},
  title     = {Theorie der Malerei},
  location  = {München},
  year      = {2008},
  series    = {Phantasos},
  number    = {8},
  shorthand = {Casanova, Theorie der Malerei},
  keywords  = {source},
}
\end{bibexample}
As Casanova's text is being published in a book, in the usual case its citation would automatically consist of the author's name and the year of the publication.
However, you will notice that the citation footnote is an exact copy of the |source| option's content:
\printbib[6em]{CasanovaMalerei}

\DescribeMacro{shorttitle}
Defining a |shorttitle| can be especially useful when encyclopediae whose title starts with an article are being cited.
%Include Grove encyclopedia!

\DescribeMacro{sortkey}
When entering (exhibiton) catalogs.

\DescribeMacro{type}
To specify the type of thesis. Possible values are |phdthesis| and |habil| for German \foreignquote{ngerman}{Habilitationen}.


\subsection{Entry types}

In this subsection we will bring the entry fields outlined above to life.

\subsubsection{\texttt{arthistory-bonn}-specific entry types}\label{sec:arth-etypes}

\DescribeMacro{@Catalog}
This entry type marks catalogs of the permanent inventory of a museum's art collection (\enquote{Bestandskatalog}).
\begin{bibexample}[label=KatSORRusche2010]{{@}catalog\{KatSORRusche2010,…\}}
@Catalog{KatSORRusche2010,
  editor   = {Hans-Joachim Raupp},
  title    = {Historien und Allegorien},
  year     = {2010},
  location = {Münster and Hamburg and London},
  keywords = {catalog},
  label    = {S{{\O}}R Rusche},
  number   = {4},
  series   = {Niederländische Malerei des 17. Jahrhunderts der S{{\O}}R Rusche-Sammlung},
  sortkey  = {Kat. S{{\O}}R Rusche 2010},
}
\end{bibexample}
\printbib[6em]{KatSORRusche2010}
Note that we defined the field |keywords| in line 9 in order to allow us to print a separate bibliography for (exhibition) catalogs later.
Moreover, we defined a |sortkey|, according to which the bibliography entry will be sorted in the respective (sub) bibliography.%
\footnote{Note that in this case the |sortkey| is actually identical to the short citation printed in the text or in the left column of the bibliography, so that we could have defined it as a |shorthand| and left out the |label| entry. However, we intend to sort automatically in a future update, rendering a |sortkey| unneccesary.}

\DescribeMacro{@Exhibcatalog}
This is for catalogs of temporary exhibitions.
\begin{bibexample}[label=AusstellungBonn2005]{{@}Exhibcatalog\{AusstellungBonn2005,…\}}
@Exhibcatalog{AusstellungBonn2005,
  editor          = {Jutta Frings},
  year            = {2005},
  location        = {Leipzig},
  eventtitle      = {Barock im Vatikan},
  eventsubtitle   = {Kunst und Kultur der Päpste II 1572--1676},
  eventdate       = {2005/2006},
  exhibfirstdate  = {2005-11-25/2006-03-19},
  exhibfirstmuseum = {Kunst- und Ausstellungshalle der Bundesrepublik Deutschland},
  exhibfirstlocation = {Bonn},
  exhibseconddate = {2006-04-12/2006-07-10},
  exhibsecondmuseum = {Martin-Gropius-Bau},
  exhibsecondlocation = {Berlin},
  keywords        = {Ausstellung},
  sortkey         = {Ausst.-Kat. Bonn/Berlin 2005},
}
\end{bibexample}
\printbib[6em]{AusstellungBonn2005}
Defining a common keyword for catalogs (line 14) allows us to print a separate subbibliography later. In order to sort the entry properly, we defined a |sortkey|. 
Note that it suffices to include but the first exhibition year in the key. The actual short citation is being generated automatically and utilizes the field |eventdate|.

The exhibition's title is listed in |eventtitle| and |eventsubtitle|, 
while time and places are defined in lines 7--13. 
As usual, the fields |year| and |location| refer to the published book itself, and not the actual exhibition.

\subsubsection{Standard \texttt{bib}\LaTeX{} types}

\DescribeMacro{@Article}
\begin{bibexample}[label=Schlegel1992]{{@}Article\{Schlegel1992,…\}}
@Article{Schlegel1992,
  author       = {Schlegel, Ursula},
  title        = {Ein Terracottamodell des Bartolomeo Ammannati},
  journaltitle = {Paragona/Arte},
  volume       = {43},
  pages        = {25--30},
  year         = {1992},
  number       = {503},
}
\end{bibexample}
\printbib[6em]{Schlegel1992}

\DescribeMacro{@Book}
Our first |book| example illustrates how to handle qualification theses along the way.
\begin{bibexample}[label=Kanz2002]{{@}Book\{Kanz2002,…\}}
@Book{Kanz2002,
  author      = {Kanz, Roland},
  title       = {Die Kunst des Capriccio},
  subtitle    = {Kreativer Eigensinn in Renaissance und Barock},
  location    = {München and Berlin},
  year        = {2002},
  series      = {Kunstwissenschaftliche Studien},
  number      = {103},
  thesisdate  = {2000},
  institution = {Düsseldorf, Univ.},
  type        = {habil},
}
\end{bibexample}
\printbib[6em]{Kanz2002}
Apart from standard fields, we defined a |thesisdate|, the |institution| at which the thesis was defended, and the thesis |type|, a \foreignquote{ngerman}{Habilitation} in this case.

While the book in example \ref{Kanz2002} is based on the submitted thesis in its entirety, the following example is only partly based on a submitted PhD thesis.
\begin{bibexample}[label=Dobler2009]{{@}Book\{Dobler2009,…\}}
@Book{Dobler2009,
  location = {München},
  title = {Die Juristenkapellen Rivaldi, Cerri und Antamoro: Form, Funktion und Intention römischer Familienkapellen im Sei- und Settecento},
  series = {Römische Studien der Bibliotheca Hertziana},
  number = {22},
  author = {Dobler, Ralph-Miklas},
  date = {2009},
  thesisdate = {2004},
  institution = {Berlin, Freie Univ.},
  type = {phdthesis},
  thesistype = {tzugl},
}
\end{bibexample}
\printbib[6em]{Dobler2009}
Note that we added the field |thesistype = {tzugl}| to reflect that the book is only partly based on a thesis.

Our third |book| example illustrates how to deal with primary sources.
\begin{bibexample}[label=PalladioArchitektur]{{@}Book\{PalladioArchitektur,…\}}
@Book{PalladioArchitektur,
  author     = {Palladio, Andrea},
  title      = {Die vier Bücher zur Architektur},
  location   = {Zürich and München},
  year       = {1993},
  edition    = {4},
  editor     = {Beyer, Andreas and Schütte, Ulrich},
  translator = {Beyer, Andreas and Schütte, Ulrich},
  shorthand  = {Palladio, Vier Bücher zur Architektur},
  keywords   = {source},
}
\end{bibexample}
\printbib[6em]{PalladioArchitektur}
First, we defined a |shorthand| to override the short citation that would have been generated otherwise. 
Second, we added a keyword so that we will be able to print a separate bibliography for primary sources later.

\DescribeMacro{@Reference}
This entry type is suited for encyclopediae.
\begin{bibexample}[label=AllgemeinesKunstlerlexikon]{{@}Book\{AllgemeinesKunstlerlexikon,…\}}
@Reference{AllgemeinesKunstlerlexikon,
  editor      = {Beyer, Andreas and others},
  editora     = {Günter Meißner},
  editoratype = {founder},
  title       = {Allgemeines Künstlerlexikon},
  subtitle    = {Die Bildenden Künstler aller Zeiten und Völker},
  date        = {1992/open},
  volume      = {1\psqq},
  note        = {Bd. 1--3 Leipzig, Bd. 4--64 München},
  publisher   = {De Gruyter},
  location    = {Berlin},
  shorthand   = {Allgemeines Künstlerlexikon},
}
\end{bibexample}
\printbib[6em]{AllgemeinesKunstlerlexikon}
Most importantly, we defined the field |shorthand| to override the standard short citation label.

Because this encyclopedia consists of a large number of volumes, it adds a couple of peculiarities. 
First, in addition to its current |editor|, its founder is listed in the fields |editora| and |editoratype|.

Next, when the field |publisher| is defined, the publishing house's name will be printed before that encyclopedia's title, 
irrespective of whether the preamble option |publisher| is enabled or not (see \cref{preamble_options}).

Third, because the publisher's location changed twice in the course of publishing past volumes, we have added a |note| that lists past locations.

\DescribeMacro{@Review}
This is for reviews of dissertation or habilitation theses, conference proceedings, other scientific publications, exhibitions etc.
For a full citation of a review it is wise to make a separate bibliography entries for the reviewed work and for the review itself.%
\footnote{The reviewed work will not be listed in the bibliography unless it is cited directly in the text.}
The following example will show an easy way to combine the review with the reviewed work:
\begin{bibexample}[label=vonEngelberg2003]{{@}Review\{vonEngelberg2003,…\}}
@Book{Brossette2002,
  author   = {Brossette, Ursula},
  title    = {Inszenierung des Sakralen},
  subtitle = {Das Theatralische Raum- und Ausstattungsprogramm süddeutscher Barockkirchen in seinem liturgischen und zeremoniellen Kontext},
  location = {Weimar},
  year     = {2002},
  series   = {Marburger Studien zur Kunst- und Kulturgeschichte},
  number   = {4},
}
@Review{vonEngelberg2003,
  author       = {von Engelberg, Meinrad},
  journaltitle = {H-ArtHist},
  related      = {Brossette2002},
  relatedtype  = {reviewof},
  arthist      = {35},
  year         = {2003},
  arthistdate  = {2003-11-12},
}\end{bibexample}
\printbib[6em]{vonEngelberg2003}
You may have noticed that the review (|vonEngelberg2003|) is connected to the entry |Brossette2002| by the field |related|.
In addition we need to qualify the relation between the connected entries:
This is done with |relatedtype={reviewof}|, which is reserved for reviews and triggers the inclusion of the translation of \enquote{Review of}, e.\,g. \enquote{Rezension von} in German, which will be printed in squared brackets.
%You don’t have to type in all relevant information of the reviewed work in the entry of the review, 
%since they will be inserted automatically and dynamically with the  |related|-function. 
%So whenever settings in the reviewed work are changed the print of the review will be automatically adjusted. 
%Furthermore, even if the review is cited, the reviewed work won't be listed in the bibliography until it is explicitly cited in the text.

The review in this example was published in the newsletter H-ArtHist, as the |journaltitle| indicates. 
We have defined two fields for H-ArtHist reviews. 
First, |arthistdate| specifies the publication day of the newsletter. 
\emph{Important:} For technical reasons, you will have to specify the field |year| or |date| as well!

Second, H-ArtHist reviews' URLs differ only by a number. 
Instead of entering the entire URL, read off the number and add it to |arthist| (here |arthist = {35}|); 
this will automatically generate the review's correct URL in the bibliography.


\section{Preamble options}\label{preamble_options}
In this section we describe options that can be loaded along with |bib|\LaTeX{} in the document preamble. 
With one exception, every option will lead to a deviation from the rules advised by the KHI first-year students' guideline; 
several options listed will allow the user to adhere to bibliography practices common in the field. 
If you do not intend to deviate at all, you can skip this section.

\DescribeMacro{allnamesfamilygiven}
When enabled, last names will precede first names in all instances.

\DescribeMacro{citeauthorformat}
You can chose how the name of authors or editors are displayed within your text when they are cited with \cs{citeauthor}\marg{bibtex-key}.
You can chose between the options \meta{initials}, \meta{full}, \meta{family}, \meta{firstfull}.

\DescribeMacro{enddot}
When including |enddot=true|, every bibliography entry will end with a dot.

\DescribeMacro{firstcitefull}
With |firstcitefull = true|, the first time (and \emph{only} the first time) a work is being cited in the document, a full citation will be printed (in a footnote).

\DescribeMacro{namelinked}
When included and |hyperref| loaded, both name and year in a short citation will link to the respective bibliography entry.

\DescribeMacro{pagesfull}
When including |pagesfull| in the options, bibliography entries' page numbers will be preceded by \enquote{pp.} (or \enquote{S.} in German). 
The same holds for citation postnotes if they contain page numbers.

\DescribeMacro{publisher}
The publisher is being listed in the bibliography entries.

\DescribeMacro{width}
|width=|\meta{value} defines the width of the left bibliography column.


\section{Cite commands}\label{cite-commands}
|arthistory-bonn| supports most/all standard |bib|\LaTeX{} citation commands. We refer the reader said package's documentation to learn more about the full set of commands. 
In the following, we will describe, for users with little experience in \LaTeX{} or |bib|\LaTeX, how standard citation commands can be employed to abide by the KHI's citation rules.

\DescribeMacro{\cite}%
The standard \cs{cite} command invokes a authoryear-style citation without any parentheses. Because of the KHI's requirements, \cs{cite} will typically be invoked from within a footnote:
\begin{code}
\footnote{
  *@\ldots@*
  \cite*@\oarg{prenote}\oarg{postnote}\marg{bibtex-key}@*
  *@\ldots@*
}
\end{code}

\meta{prenote} sets a short preliminary note (e.\,g. \enquote{Vgl.}) and \meta{postnote} is usually used for page numbers.
If only one optional argument is used then it is \oarg{postnote}.
\begin{code}
\footnote{*@\ldots@*\cite*@\oarg{postnote}\marg{bibtex-key}\ldots@*}
\end{code}
The \meta{bibtex-key} corresponds to the key from the bibliography file.

\DescribeMacro{\footcite}
The same as manually adding a footnote first and \cs{cite} subsequently can be achieved in one step via the \cs{footcite} command:
\begin{code}
\footcite*@\oarg{prenote}\oarg{postnote}\marg{bibtex-key}%@*
\end{code}
This command will be useful if nothing more than a citation with very short prenotes and/or postnotes is needed. 
When a citation is embedded in a text paragraph, the former combination of \cs{footnote} and \cs{cite} is advisable.

%As noted above, all the well-documented citation commands of the |bib|\LaTeX{} package are supported.
\DescribeMacro{\cites}
If one wants to cite several authors or works at once, a very convenient way is the following, using the \cs{cites}-command (typically in a footnote):
\begin{code}
\cites(pre-prenote)(post-postnote)
  *@\oarg{prenote}\oarg{postnote}\marg{bibtex-key}@*%
  *@\oarg{prenote}\oarg{postnote}\marg{bibtex-key}@*%
  *@\oarg{prenote}\oarg{postnote}\marg{bibtex-key}\ldots@*
\end{code}
Other examples are \cs{parencite},\cs{textcite} and their multi-entry alternatives, and commands such as \cs{citeauthor} and \cs{citetitle}.

\DescribeMacro{smartcite}\DescribeMacro{autocite}
Note that \cs{smartcite} and \cs{autocite} behave a little bit differently than in \enquote{standard} |bib|\LaTeX{} styles. 
When appearing in a footnote, both commands will behave as |arthistory-bonn|'s \cs{cite} rather than \cs{parencite}. 
In addition to that, by default \cs{autocite} appearing in the text body behaves like \cs{footcite}.


\section{Separate bibliographies}\label{sec:sepbib}
Here, we describe how you can list separate bibliographies for primary sources and secondary literature (and possibly catalogs as well). This can be achieved by standard |bib|\LaTeX{} procedures; experienced users will want to skip this section.

You may have noticed that we listed the option |keywords = {source}| in \cref{sec:bibl-efields} and that we used it in \cref{sec:arth-etypes} when describing the entry type |@Exhibcatalog|. The sole purpose for this is to prepare listing a separate bibliography for primary sources.

If you would like to include a separate bibliography for (exhibition) catalogs as well, each catalog entry must contain a common keyword such as |keywords = {catalog}|.

To include separate bibliographies in the document, instead of typing |printbibliography|, include
\begin{code}
\printbibheading[%
  heading=bibliography,%
  title={Bibliography}]
\printbibliography[%
  keyword=source,%
  heading=subbibliography,
  title={Primary sources}]
\printbibliography[%
  notkeyword=catalog,%
  notkeyword=source,%
  heading=subbibliography,%
  title={Secondary literature}]
\printbibliography[%
  keyword=catalog,%
  heading=subbibliography,%
  title={Exhibition catalogs}]
\end{code}
As you will notice, the first lines specify a heading for the bibliography as a whole. The next four lines make sure that a subbibliography be printed that lists \emph{only} those entries that include the keyword \enquote{source} (and nothing else). The next four lines lead to printing a subbibliography for all bibliography entries that do not have the keywords \enquote{source} or \enquote{catalog} -- hence, everything we would like to call standard secondary literature. The last block is for catalogs.

Note: If you include entries of the type |@InExhibcatalog| that are linked to an |@Exhibcatalog| via a |crossref| field, this article will \enquote*{inherit} its parent catalog's keyword and be listed in the catalogs' subbibliography. 
If you do not want this, simply add something like |keywords = {InCatalog}| to the |@InExhibcatalog| entry. 
This will have the article listed in the secondary literature subbibliography.

%\DescribeMacro{\parencite}
%Sometimes a citation has to be put in parentheses. 
%Therefore we implemented the command \cs{parencite}:
%\begin{code}
%\parencite*@\oarg{postnote}\marg{bibtex-key}%@*
%\end{code} 
%This cite command takes care of the correct corresponding parentheses and brackets.
%Especially in |@Inreference| citations the parentheses  change to (square) brackets.
%
%
%\DescribeMacro{\parencites}
%Of course there is also the possibility to cite several authors/works in parentheses.
%This is done with \cs{parencites}:
%\begin{code}
%\parencites(pre-prenote)(post-postnote)%
%*@\oarg{prenote}\oarg{postnote}\marg{bibtex-key}@*%
%*@\oarg{prenote}\oarg{postnote}\marg{bibtex-key}@*%
%*@\oarg{prenote}\oarg{postnote}\marg{bibtex-key}\ldots@*
%\end{code}
 %
%\DescribeMacro{\textcite}
%Beside the listed \cs{cite} commands above there is a third way of citing:
%\cs{textcite} is useful if the author should be mentioned in the text and
%the remaining components such as year and page will immediately follow in parentheses. 
%\begin{code}
%\textcite*@\oarg{postnote}\marg{bibtex-key}@*
%\end{code} 
%
%\DescribeMacro{\textcites}
%And again there is also a \cs{textcites} in case of several authors: 
%\begin{code}
%\textcites(pre-prenote)(post-postnote)%
  %*@\oarg{prenote}\oarg{postnote}\marg{bibtex-key}@*%
  %*@\oarg{prenote}\oarg{postnote}\marg{bibtex-key}@*%
  %*@\oarg{prenote}\oarg{postnote}\marg{bibtex-key}\ldots@*
%\end{code}
%
%
%
%Futhermore you can use following \cs{cite} commands:
%\DescribeMacro{\footcite} 
%\begin{code}
%\footcite*@\oarg{prenote}\oarg{postnote}\marg{bibtex-key}@*
%\end{code} 
%
%\begin{example}
%\footcite{Schlegel1992}
%\end{example}
%
%
%\DescribeMacro{\footcitetext} 
%\begin{code}
%\footcitetext*@\oarg{prenote}\oarg{postnote}\marg{bibtex-key}@*
%\end{code} 
%
%\begin{example}
%\footcitetext{Schlegel1992}
%\end{example}
%
%
%\DescribeMacro{\smartcite} 
%\begin{code}
%\smartcite*@\oarg{prenote}\oarg{postnote}\marg{bibtex-key}@*
%\end{code} 
%
%\begin{example}
%\smartcite{Schlegel1992}
%\end{example}
%
%
%
%
%
%
%\DescribeMacro{\citeauthor}\DescribeMacro{\citetitle}\label{citeauthor}%
%Furthermore and in addition to the ›normal‹ \cs{cite}-commands one can also cite only the author or the work title in the text and in the footnotes.
%\begin{code}
%\citeauthor*@\oarg{prenote}\oarg{postnote}\marg{bibtex-key}@*
%\end{code} 
  %and for the works 
%\begin{code}
%\citetitle*@\oarg{prenote}\oarg{postnote}\marg{bibtex-key}@*
%\end{code} 


\end{document}